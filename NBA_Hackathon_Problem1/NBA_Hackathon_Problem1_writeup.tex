\documentclass{article}\usepackage[]{graphicx}\usepackage[]{color}
%% maxwidth is the original width if it is less than linewidth
%% otherwise use linewidth (to make sure the graphics do not exceed the margin)
\makeatletter
\def\maxwidth{ %
  \ifdim\Gin@nat@width>\linewidth
    \linewidth
  \else
    \Gin@nat@width
  \fi
}
\makeatother

\definecolor{fgcolor}{rgb}{0.345, 0.345, 0.345}
\newcommand{\hlnum}[1]{\textcolor[rgb]{0.686,0.059,0.569}{#1}}%
\newcommand{\hlstr}[1]{\textcolor[rgb]{0.192,0.494,0.8}{#1}}%
\newcommand{\hlcom}[1]{\textcolor[rgb]{0.678,0.584,0.686}{\textit{#1}}}%
\newcommand{\hlopt}[1]{\textcolor[rgb]{0,0,0}{#1}}%
\newcommand{\hlstd}[1]{\textcolor[rgb]{0.345,0.345,0.345}{#1}}%
\newcommand{\hlkwa}[1]{\textcolor[rgb]{0.161,0.373,0.58}{\textbf{#1}}}%
\newcommand{\hlkwb}[1]{\textcolor[rgb]{0.69,0.353,0.396}{#1}}%
\newcommand{\hlkwc}[1]{\textcolor[rgb]{0.333,0.667,0.333}{#1}}%
\newcommand{\hlkwd}[1]{\textcolor[rgb]{0.737,0.353,0.396}{\textbf{#1}}}%
\let\hlipl\hlkwb

\usepackage{framed}
\makeatletter
\newenvironment{kframe}{%
 \def\at@end@of@kframe{}%
 \ifinner\ifhmode%
  \def\at@end@of@kframe{\end{minipage}}%
  \begin{minipage}{\columnwidth}%
 \fi\fi%
 \def\FrameCommand##1{\hskip\@totalleftmargin \hskip-\fboxsep
 \colorbox{shadecolor}{##1}\hskip-\fboxsep
     % There is no \\@totalrightmargin, so:
     \hskip-\linewidth \hskip-\@totalleftmargin \hskip\columnwidth}%
 \MakeFramed {\advance\hsize-\width
   \@totalleftmargin\z@ \linewidth\hsize
   \@setminipage}}%
 {\par\unskip\endMakeFramed%
 \at@end@of@kframe}
\makeatother

\definecolor{shadecolor}{rgb}{.97, .97, .97}
\definecolor{messagecolor}{rgb}{0, 0, 0}
\definecolor{warningcolor}{rgb}{1, 0, 1}
\definecolor{errorcolor}{rgb}{1, 0, 0}
\newenvironment{knitrout}{}{} % an empty environment to be redefined in TeX

\usepackage{alltt}
%packages
\usepackage[sc]{mathpazo}
\usepackage[T1]{fontenc}
\usepackage{geometry}
\geometry{verbose,tmargin=2.5cm,bmargin=2.5cm,lmargin=2.5cm,rmargin=2.5cm}
\setcounter{secnumdepth}{2}
\setcounter{tocdepth}{2}
\usepackage{url}
\usepackage[unicode=true,pdfusetitle,
 bookmarks=true,bookmarksnumbered=true,bookmarksopen=true,bookmarksopenlevel=2,
 breaklinks=false,pdfborder={0 0 1},backref=false,colorlinks=false]
 {hyperref}
\hypersetup{pdfstartview={XYZ null null 1}}
\usepackage{breakurl}
\usepackage{/Users/zeyu/Dropbox/misc/academic_misc/LATEX/mylecstyles}
\IfFileExists{upquote.sty}{\usepackage{upquote}}{}
\begin{document}



\title{NBA Hackathon Problem 1}
\author{Chia-Wei Hsu, Zeyu Li}
\maketitle

\section*{(a)}
\section*{Approach 1: Exact answer}

    
The exact probability that the Warriors would never lose consecutive games at any
point during an 82-game season is as follows, 

\begin{equation*}
    \sum_{x = 41}^{82}\binom{x+1}{82-x}0.8^x0.2^{82-x}
\end{equation*}

Above sum is approximately 0.0588169.

\section*{Approach 2: Simulation}



(Note: code for simulation see separate attached file). We build a simulator
that simulates many 82-game seasons. Each 82-game season is binomially
distributed with a success probability of 0.8 ($p = 0.8$) and a failure
probability of 0.2 ($q = 1 - p = 0.2$). We draw random samples of size 100 from
the binomial distribution and count the number of samples where the Warriors
does not lose consecutive games at any point in a given 82-game season. This way
we obtain an approximated probability. We repeat this process $n$ times (we
choose $n$ to be $1000$ to ensure that law of large numbers applies). After
repeating for $1000$ times, we obtain an averaged probability of 0.05913 and a
standard error (variability) of 0.0236876. We are $95\%$ confidence that the
true probability lies within the interval $(0.0544872, \ 0.0637728)$.
@

\section*{(b)}
The null hypothesis is that the Warriors would never lose consecutive games at
any point during an 82-game season, that is $\Prob(no\ consecutive\ losses) =
100\%$. From above simulations, we can calculate a $95\%$ confidence interval,
which is $(5.4487223\%, \ 6.3772777\%)$. Since it does not contain
$100\%$, we reject the null hypothesis. 


\section*{(c)}
This question is easy to derive from the exact solution in the form presented in
(a). The answer is calculated to be $90.3772241\%$.


\end{document}

